\documentclass[a3paper]{article}
\usepackage[T1]{fontenc}
\usepackage{lmodern}
\usepackage{amssymb,amsmath}
\usepackage{ifxetex,ifluatex}
\usepackage{fixltx2e} % provides \textsubscript
% use upquote if available, for straight quotes in verbatim environments
\IfFileExists{upquote.sty}{\usepackage{upquote}}{}
\ifnum 0\ifxetex 1\fi\ifluatex 1\fi=0 % if pdftex
  \usepackage[utf8]{inputenc}
\else % if luatex or xelatex
  \ifxetex
    \usepackage{mathspec}
    \usepackage{xltxtra,xunicode}
  \else
    \usepackage{fontspec}
  \fi
  \defaultfontfeatures{Mapping=tex-text,Scale=MatchLowercase}
  \newcommand{\euro}{€}
\fi
% use microtype if available
\IfFileExists{microtype.sty}{\usepackage{microtype}}{}
\usepackage[margin=1in]{geometry}
\usepackage{color}
\usepackage{fancyvrb}
\newcommand{\VerbBar}{|}
\newcommand{\VERB}{\Verb[commandchars=\\\{\}]}
\DefineVerbatimEnvironment{Highlighting}{Verbatim}{commandchars=\\\{\}}
% Add ',fontsize=\small' for more characters per line
\usepackage{framed}
\definecolor{shadecolor}{RGB}{248,248,248}
\newenvironment{Shaded}{\begin{snugshade}}{\end{snugshade}}
\newcommand{\KeywordTok}[1]{\textcolor[rgb]{0.13,0.29,0.53}{\textbf{{#1}}}}
\newcommand{\DataTypeTok}[1]{\textcolor[rgb]{0.13,0.29,0.53}{{#1}}}
\newcommand{\DecValTok}[1]{\textcolor[rgb]{0.00,0.00,0.81}{{#1}}}
\newcommand{\BaseNTok}[1]{\textcolor[rgb]{0.00,0.00,0.81}{{#1}}}
\newcommand{\FloatTok}[1]{\textcolor[rgb]{0.00,0.00,0.81}{{#1}}}
\newcommand{\CharTok}[1]{\textcolor[rgb]{0.31,0.60,0.02}{{#1}}}
\newcommand{\StringTok}[1]{\textcolor[rgb]{0.31,0.60,0.02}{{#1}}}
\newcommand{\CommentTok}[1]{\textcolor[rgb]{0.56,0.35,0.01}{\textit{{#1}}}}
\newcommand{\OtherTok}[1]{\textcolor[rgb]{0.56,0.35,0.01}{{#1}}}
\newcommand{\AlertTok}[1]{\textcolor[rgb]{0.94,0.16,0.16}{{#1}}}
\newcommand{\FunctionTok}[1]{\textcolor[rgb]{0.00,0.00,0.00}{{#1}}}
\newcommand{\RegionMarkerTok}[1]{{#1}}
\newcommand{\ErrorTok}[1]{\textbf{{#1}}}
\newcommand{\NormalTok}[1]{{#1}}
\usepackage{graphicx}
% Redefine \includegraphics so that, unless explicit options are
% given, the image width will not exceed the width of the page.
% Images get their normal width if they fit onto the page, but
% are scaled down if they would overflow the margins.
\makeatletter
\def\ScaleIfNeeded{%
  \ifdim\Gin@nat@width>\linewidth
    \linewidth
  \else
    \Gin@nat@width
  \fi
}
\makeatother
\let\Oldincludegraphics\includegraphics
{%
 \catcode`\@=11\relax%
 \gdef\includegraphics{\@ifnextchar[{\Oldincludegraphics}{\Oldincludegraphics[width=\ScaleIfNeeded]}}%
}%
\ifxetex
  \usepackage[setpagesize=false, % page size defined by xetex
              unicode=false, % unicode breaks when used with xetex
              xetex]{hyperref}
\else
  \usepackage[unicode=true]{hyperref}
\fi
\hypersetup{breaklinks=true,
            bookmarks=true,
            pdfauthor={},
            pdftitle={},
            colorlinks=true,
            citecolor=blue,
            urlcolor=blue,
            linkcolor=magenta,
            pdfborder={0 0 0}}
\urlstyle{same}  % don't use monospace font for urls
\setlength{\parindent}{0pt}
\setlength{\parskip}{6pt plus 2pt minus 1pt}
\setlength{\emergencystretch}{3em}  % prevent overfull lines
\setcounter{secnumdepth}{0}

\author{}
\date{}

\begin{document}

\begin{center}
\normalsize
\end{center}


\section{Motor Trends : Automatic or Manual transmission for better
mileage
?}\label{motor-trends-automatic-or-manual-transmission-for-better-mileage}

\textbf{by P. Paquay}

\subsection{Executive summary}\label{executive-summary}

In this report we try to answer the question : ``Is automatic or manual
transmission better for mpg ?''. To answer this question we used a
dataset from the 1974 Motor Trend US magazine, and ran some statistical
tests and a regression analysis. On one hand the statistical tests show
(without controlling for other car design features) a difference in mean
of about 7 miles more for the manual transmitted cars. On the other
hand, the regression analysis indicate that by taking into account other
variables like horse power and weight, manual transmitted cars are only
1.8 miles better than automatic transmitted cars and also that this
result is not significant. So to get a better mileage it's probably
better to consider cars of a certain weight and horse power than to
consider manual or automatic transmission.

\subsection{Cleaning data}\label{cleaning-data}

The first step of our analysis is simply to load and take a look at the
data.

\begin{Shaded}
\begin{Highlighting}[]
\KeywordTok{data}\NormalTok{(mtcars)}
\KeywordTok{str}\NormalTok{(mtcars)}
\end{Highlighting}
\end{Shaded}

\begin{verbatim}
## 'data.frame':    32 obs. of  11 variables:
##  $ mpg : num  21 21 22.8 21.4 18.7 18.1 14.3 24.4 22.8 19.2 ...
##  $ cyl : num  6 6 4 6 8 6 8 4 4 6 ...
##  $ disp: num  160 160 108 258 360 ...
##  $ hp  : num  110 110 93 110 175 105 245 62 95 123 ...
##  $ drat: num  3.9 3.9 3.85 3.08 3.15 2.76 3.21 3.69 3.92 3.92 ...
##  $ wt  : num  2.62 2.88 2.32 3.21 3.44 ...
##  $ qsec: num  16.5 17 18.6 19.4 17 ...
##  $ vs  : num  0 0 1 1 0 1 0 1 1 1 ...
##  $ am  : num  1 1 1 0 0 0 0 0 0 0 ...
##  $ gear: num  4 4 4 3 3 3 3 4 4 4 ...
##  $ carb: num  4 4 1 1 2 1 4 2 2 4 ...
\end{verbatim}

Now we coerce the ``cyl'', ``vs'', ``gear'', ``carb'' and ``am''
variables into factor variables.

\begin{Shaded}
\begin{Highlighting}[]
\NormalTok{mtcars$cyl <-}\StringTok{ }\KeywordTok{factor}\NormalTok{(mtcars$cyl)}
\NormalTok{mtcars$vs <-}\StringTok{ }\KeywordTok{factor}\NormalTok{(mtcars$vs)}
\NormalTok{mtcars$gear <-}\StringTok{ }\KeywordTok{factor}\NormalTok{(mtcars$gear)}
\NormalTok{mtcars$carb <-}\StringTok{ }\KeywordTok{factor}\NormalTok{(mtcars$carb)}
\NormalTok{mtcars$am <-}\StringTok{ }\KeywordTok{factor}\NormalTok{(mtcars$am)}
\end{Highlighting}
\end{Shaded}

For a better readability, we rename the levels of the ``am'' variable
into ``Auto'' and ``Manual''.

\begin{Shaded}
\begin{Highlighting}[]
\KeywordTok{levels}\NormalTok{(mtcars$am) <-}\StringTok{ }\KeywordTok{c}\NormalTok{(}\StringTok{"Auto"}\NormalTok{, }\StringTok{"Manual"}\NormalTok{)}
\end{Highlighting}
\end{Shaded}

\subsection{Graphics}\label{graphics}

We begin by plotting boxplots of the variable ``mpg'' when ``am'' is
``Auto'' or ``Manual'' (see Figure 1 in the appendix). This plot hints
at an increase in mpg when gearing was manual but this data may have
other variables which may play a bigger role in determination of mpg.

We then plot the relationships between all the variables of the dataset
(see Figure 2 in the appendix). We may note that variables like ``cyl'',
``disp'', ``hp'', ``drat'', ``wt'', ``vs'' and ``am'' seem highly
correlated to ``mpg''.

\subsection{Inference}\label{inference}

We may also run a some tests to compare the mpg means between automatic
and manual transmissions.

\subsubsection{T-test}\label{t-test}

We begin by using a t-test assuming that the mileage data has a normal
distribution.

\begin{Shaded}
\begin{Highlighting}[]
\KeywordTok{t.test}\NormalTok{(mpg ~}\StringTok{ }\NormalTok{am, }\DataTypeTok{data =} \NormalTok{mtcars)}
\end{Highlighting}
\end{Shaded}

\begin{verbatim}
## 
##  Welch Two Sample t-test
## 
## data:  mpg by am
## t = -3.767, df = 18.33, p-value = 0.001374
## alternative hypothesis: true difference in means is not equal to 0
## 95 percent confidence interval:
##  -11.28  -3.21
## sample estimates:
##   mean in group Auto mean in group Manual 
##                17.15                24.39
\end{verbatim}

The test results clearly shows that the manual and automatic
transmissions are significatively different.

\subsubsection{Wilcoxon test}\label{wilcoxon-test}

Next we use a nonparametric test to determine if there's a difference in
the population means.

\begin{Shaded}
\begin{Highlighting}[]
\KeywordTok{wilcox.test}\NormalTok{(mpg ~}\StringTok{ }\NormalTok{am, }\DataTypeTok{data =} \NormalTok{mtcars)}
\end{Highlighting}
\end{Shaded}

\begin{verbatim}
## Warning: cannot compute exact p-value with ties
\end{verbatim}

\begin{verbatim}
## 
##  Wilcoxon rank sum test with continuity correction
## 
## data:  mpg by am
## W = 42, p-value = 0.001871
## alternative hypothesis: true location shift is not equal to 0
\end{verbatim}

The Wilcoxon test also rejects the null hypothesis that the mileage data
of the manual and automatic transmissions are from the same population
(indicating a difference).

\subsection{Regression analysis}\label{regression-analysis}

First we need to select a model, we proceed by using the Akaike
Information Criteria (AIC) in a stepwise algorithm. This algorithm does
not evaluate the AIC for all possible models but uses a search method
that compares models sequentially. Thus it bears some comparison to the
classical stepwise method but with the advantage that no dubious
p-values are used.

\begin{Shaded}
\begin{Highlighting}[]
\NormalTok{model.all <-}\StringTok{ }\KeywordTok{lm}\NormalTok{(mpg ~}\StringTok{ }\NormalTok{., }\DataTypeTok{data =} \NormalTok{mtcars)}
\NormalTok{model <-}\StringTok{ }\KeywordTok{step}\NormalTok{(model.all)}
\end{Highlighting}
\end{Shaded}

\begin{Shaded}
\begin{Highlighting}[]
\KeywordTok{summary}\NormalTok{(model)}
\end{Highlighting}
\end{Shaded}

\begin{verbatim}
## 
## Call:
## lm(formula = mpg ~ cyl + hp + wt + am, data = mtcars)
## 
## Residuals:
##    Min     1Q Median     3Q    Max 
## -3.939 -1.256 -0.401  1.125  5.051 
## 
## Coefficients:
##             Estimate Std. Error t value Pr(>|t|)    
## (Intercept)  33.7083     2.6049   12.94  7.7e-13 ***
## cyl6         -3.0313     1.4073   -2.15   0.0407 *  
## cyl8         -2.1637     2.2843   -0.95   0.3523    
## hp           -0.0321     0.0137   -2.35   0.0269 *  
## wt           -2.4968     0.8856   -2.82   0.0091 ** 
## amManual      1.8092     1.3963    1.30   0.2065    
## ---
## Signif. codes:  0 '***' 0.001 '**' 0.01 '*' 0.05 '.' 0.1 ' ' 1
## 
## Residual standard error: 2.41 on 26 degrees of freedom
## Multiple R-squared:  0.866,  Adjusted R-squared:  0.84 
## F-statistic: 33.6 on 5 and 26 DF,  p-value: 1.51e-10
\end{verbatim}

The AIC algorithm tells us to consider ``cyl'', ``wt'' and ``hp'' as
confounding variables. The individual p-values allows us to reject the
hypothesis that the coefficients are null. The adjusted r-squared is
0.8401, so we may conclude that more than 84\% of the variation is
explained by the model.

\begin{Shaded}
\begin{Highlighting}[]
\NormalTok{model0 <-}\StringTok{ }\KeywordTok{lm}\NormalTok{(mpg ~}\StringTok{ }\NormalTok{am, }\DataTypeTok{data =} \NormalTok{mtcars)}
\KeywordTok{anova}\NormalTok{(model0, model)}
\end{Highlighting}
\end{Shaded}

\begin{verbatim}
## Analysis of Variance Table
## 
## Model 1: mpg ~ am
## Model 2: mpg ~ cyl + hp + wt + am
##   Res.Df RSS Df Sum of Sq    F  Pr(>F)    
## 1     30 721                              
## 2     26 151  4       570 24.5 1.7e-08 ***
## ---
## Signif. codes:  0 '***' 0.001 '**' 0.01 '*' 0.05 '.' 0.1 ' ' 1
\end{verbatim}

We may notice that when we compare the model with only ``am'' as
independant variable and our chosen model, we reject the null hypothesis
that the variables ``cyl'', ``hp'' and ``wt'' don't contribute to the
accuracy of the model.

The regression suggests that, other variables remaining constant, manual
transmitted cars can drive 1.8092 more miles per gallon than automatic
transmitted cars, and the results are not statistically significant.

\subsection{Residuals and diagnostics}\label{residuals-and-diagnostics}

\subsubsection{Residual analysis}\label{residual-analysis}

We begin by studying the residual plots (see Figure 3 in the appendix).
These plots allow us to verify some assumptions made before.

\begin{enumerate}
\def\labelenumi{\arabic{enumi}.}
\setcounter{enumi}{2}
\itemsep1pt\parskip0pt\parsep0pt
\item
  The Residuals vs Fitted plot seem to verify the independance
  assumption as the points are randomly scattered on the plot.
\item
  The Normal Q-Q plot seem to indicate that the residuals are normally
  distributed as the points hug the line closely.
\item
  The Scale-Location plot seem to verify the constant variance
  assumption as the points fall in a constant band.
\end{enumerate}

\subsubsection{Leverages}\label{leverages}

We begin by computing the leverages for the ``mtcars'' dataset.

\begin{Shaded}
\begin{Highlighting}[]
\NormalTok{leverage <-}\StringTok{ }\KeywordTok{hatvalues}\NormalTok{(model)}
\end{Highlighting}
\end{Shaded}

Are any of the observations in the dataset outliers ? We find the
outliers by selecting the observations with a hatvalue \textgreater{}
0.5.

\begin{Shaded}
\begin{Highlighting}[]
\NormalTok{leverage[}\KeywordTok{which}\NormalTok{(leverage >}\StringTok{ }\FloatTok{0.5}\NormalTok{)]}
\end{Highlighting}
\end{Shaded}

\begin{verbatim}
## named numeric(0)
\end{verbatim}

\subsubsection{Dfbetas}\label{dfbetas}

Next we look at the Dfbetas of the observations.

\begin{Shaded}
\begin{Highlighting}[]
\NormalTok{influential <-}\StringTok{ }\KeywordTok{dfbetas}\NormalTok{(model)}
\end{Highlighting}
\end{Shaded}

Are any of the observations in the dataset influential ? We find the
influential observations by selecting the ones with a dfbeta
\textgreater{} 1 in magnitude.

\begin{Shaded}
\begin{Highlighting}[]
\NormalTok{influential[}\KeywordTok{which}\NormalTok{(}\KeywordTok{abs}\NormalTok{(influential) >}\StringTok{ }\DecValTok{1}\NormalTok{)]}
\end{Highlighting}
\end{Shaded}

\begin{verbatim}
## numeric(0)
\end{verbatim}

\subsection{Appendix}\label{appendix}

\subsubsection{Figure 1 : Boxplots of ``mpg'' vs.
``am''}\label{figure-1-boxplots-of-mpg-vs.-am}

\begin{Shaded}
\begin{Highlighting}[]
\KeywordTok{plot}\NormalTok{(mpg ~}\StringTok{ }\NormalTok{am, }\DataTypeTok{data =} \NormalTok{mtcars, }\DataTypeTok{main =} \StringTok{"Mpg by transmission type"}\NormalTok{, }\DataTypeTok{xlab =} \StringTok{"Transmission type"}\NormalTok{, }\DataTypeTok{ylab =} \StringTok{"Miles per gallon"}\NormalTok{)}
\end{Highlighting}
\end{Shaded}

\includegraphics{./Report_files/figure-latex/unnamed-chunk-13.pdf}

\subsubsection{Figure 2 : Pairs graph}\label{figure-2-pairs-graph}

\begin{Shaded}
\begin{Highlighting}[]
\KeywordTok{pairs}\NormalTok{(mtcars, }\DataTypeTok{panel =} \NormalTok{panel.smooth, }\DataTypeTok{main =} \StringTok{"Pairs graph for MTCars"}\NormalTok{)}
\end{Highlighting}
\end{Shaded}

\includegraphics{./Report_files/figure-latex/unnamed-chunk-14.pdf}

\subsubsection{Figure 3 : Residual plots}\label{figure-3-residual-plots}

\begin{Shaded}
\begin{Highlighting}[]
\KeywordTok{par}\NormalTok{(}\DataTypeTok{mfrow =} \KeywordTok{c}\NormalTok{(}\DecValTok{2}\NormalTok{, }\DecValTok{2}\NormalTok{))}
\KeywordTok{plot}\NormalTok{(model)}
\end{Highlighting}
\end{Shaded}

\includegraphics{./Report_files/figure-latex/unnamed-chunk-15.pdf}

\end{document}
